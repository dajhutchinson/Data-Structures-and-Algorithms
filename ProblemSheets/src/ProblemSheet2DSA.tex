\documentclass[11pt,a4paper]{article}

\usepackage[margin=1in, paperwidth=8.3in, paperheight=11.7in]{geometry}
\usepackage{amsfonts}
\usepackage{amsmath}
\usepackage{enumerate}
\usepackage{enumitem}
\usepackage{fancyhdr}
\usepackage{listings}
\usepackage{stmaryrd}
\usepackage[table]{xcolor}

\begin{document}

\pagestyle{fancy}
\setlength\parindent{0pt}
\allowdisplaybreaks

% Counters
\newcounter{question}
\newcounter{qpart}[question]
\newcounter{spart}[question]
% commands
\newcommand{\nats}{\mathbb{N}}
\newcommand{\real}{\mathbb{R}}
\newcommand{\newquestion} {\stepcounter{question}}
\newcommand{\newqpart} {\stepcounter{qpart}}
\newcommand{\newspart} {\stepcounter{spart}}
\newcommand{\question}[2] {\newquestion \ifquestions \textbf{Question \arabic{question} - #1}\\ #2\\ \fi}
\newcommand{\qpart}[1] {\newqpart \ifquestions \textbf{Question \arabic{question}.\arabic{qpart}}\\ #1\\ \fi}
\newcommand{\solution}[1] {\ifsolutions\textbf{My Solution \arabic{question}}\\ #1\\ \fi}
\newcommand{\spart}[1] {\newspart\ifsolutions\textbf{My Solution \arabic{question}.\arabic{spart}}\\ #1\\ \fi}
\newcommand{\doubleplus} {+\kern-1.3ex+\kern0.8ex}
\renewcommand{\headrulewidth}{0pt}

% enviroments
\lstnewenvironment{code}
  {\lstset{mathescape=true,xleftmargin=.1\textwidth}}
  {}

% if
\newif\ifquestions
\questionstrue
\questionsfalse
\newif\ifsolutions
\solutionstrue
%\solutionsfalse

% Cover page title
\title{Data Structures \& Algorithms - Problem Sheet 2}
\author{Dom Hutchinson}
\date{\today}
\maketitle

% Header
\fancyhead[L]{Dom Hutchinson}
\fancyhead[C]{Data Structures \& Algorithms - Problem Sheet 2}
\fancyhead[R]{\today}

\question{My, how you've grown}{
This question is about the growth of functions. List these functions of $n$ in an order such that $f(n)\in O(g(n))$ whenever $f(n)$ comes before $g(n)$ in the list.\\
$$3\log_2n,\ 10^100,\ 3(\log_2n)^5,\ \sqrt{5},\ n!-14,\ n\log_2n,\ n^4-10n^2,\ n3^n,\ n2^n+n, n^2$$
}

\solution{
$$10^{100},\ 3\log_2n,\ \sqrt{n},\ n\log_2n,\ n^2,\ 3(\log_2n)^5,\ n^4-10n^2,\ n2^n+n,\ n3^n,\ n!-14$$
}

\question{Albanian Ponzi}{
(This question is loosely based on real events in Albania in 1997).\\
A company has launched the following exclusive investment scheme, which is
only open to Albanians. To take part, you invest 40, 000 Leks (equivalent to
about 280 UK pounds), and recruit 2 more Albanians to the scheme in the week
after you join, and 3 more in the week after that. At the end of the month you
will receive the million Leks invested by the 25 people who have been recruited
by the 5 people you have recruited.\\
This scheme is so exciting that in addition to the Albanians who are recruited
to the scheme by other investors, 1000 Albanians will join each week without
being recruited by another investor.\\
Write down a recurrence relation for the number of new investors in the scheme
in week n, and solve it.\\
The population of Albania is under 3 million. Show that the investment scheme
will collapse in under 10 weeks.
}

\solution{
\[\begin{array}{rrcl}
&f(0)&=&0\\
&f(1)&=&1000\\
&f(n)&=&2f(n-1)+3f(n-2)+1000\\
\mathrm{Set}&\lambda^2&=&2\lambda+3\\
\implies&\lambda^2-2\lambda-3&=&0\\
&(\lambda-3)(\lambda+1)&=&0\\
\implies&\lambda_1=-1&\&&\lambda_2=3\\
\mathrm{Set}&f(n)&=&\alpha(-1)^n+\beta3^n+\gamma\\
\\\mathrm{Since}&f(n)&=&2f(n-1)+3f(n-2)+1000\\
&&=&2[\alpha(-1)^{n-1}+\beta3^{n-1}+\gamma]+3[\alpha(-1)^{n-2}+\beta3^{n-2}+\gamma]+1000\\
&&=&2\alpha(-1)^{n-1}+2\beta3^{n-1}+2\gamma-3\alpha(-1)^{n-1}+\beta3^{n-1}+3\gamma+1000\\
&&=&-\alpha(-1)^{n-1}+\beta3^n+5\gamma+1000\\
&&=&\alpha(-1)^n+\beta3^n+5\gamma+1000\\
&&=&f(n)+4\gamma+1000\\
\implies&\gamma&=&\underline{-250}\\
\\&\mathrm{From\ conditions}\\
&f(0)&=&0\\
\implies&\alpha+\beta-250&=&0\\
\implies&\beta&=&250-\alpha\\
&f(1)&=&1000\\
\implies&-\alpha+3\beta-250&=&1000\\
\implies&-\alpha+3(250-\alpha)&=&1250\\
\implies&-4\alpha&=&500\\
\implies&\alpha&=&\underline{-125}\\
\implies&\beta&=&250-(-125)\\
&&=&\underline{375}\\
\\\mathrm{So}&f(n)&=&125(-1)^{n+1}+375(3^n)-250\\
\end{array}\]
When $n=9$, $f(9)$ is greater than the population of Albanian. 
}

\question{Algorithm Choice}{
Your colleague is trying to decide between two divide-and-conquer algorithms
for processing n data items, where n is large. She would like to use an algorithm
that scales well. The first algorithm splits the data into ten smaller sets of equal
size (or, as close as to equal size as possible if n is not divisible by 10), then
recursively runs the algorithm on each of the smaller sets in turn, and combines
these answers. The second algorithm splits the data into two sets, one with
b9n/10c elements and one with $\lceil\frac{n}{10}\rceil$ elements, recursively runs the algorithm
on each of the two smaller sets, and combines the answers. The splitting and
combining of answers takes $100n$ ms for the first algorithm and $\frac{n\sqrt{n}}{100}$ ms for the second.\\
Advise her which algorithm to use. Justify your answer using this result men-
tioned in a lecture.
}

\solution{
\[\begin{array}{rcl}
T_1(n)&=&10T_1(\lceil\frac{n}{10}\rceil)+100n\\
T_2(n)&=&T_2(\lfloor\frac{9n}{10}\rfloor)+T_2(\lceil\frac{n}{10}\rceil)+\frac{n^{3/2}}{100}\\
\end{array}\]
Analysing $T_1$ with the \textit{Master Theorem}.
$$a=10,\ b=10,\ f(n)\in O(n)\implies c=1$$
Let $p=log_ba \implies p=log_{10}10=1=c\implies T_1\in\Theta(n\log_2n)$.\\
Analysing $T_2$ with the \textit{Akra-Bazzi Formula}.
$$k=2,\ a_1=1,\ a_2=1,\ d_1=\frac{9}{10},\ d_2=\frac{1}{100},\ \alpha=100,\ f(n)\in O(n^{3/2})\implies c=\frac{3}{2}$$
Set
$$\sum_{i=1}^ka_id_i^p=1\implies\left(\frac{9}{10}\right)^p+\left(\frac{1}{10}\right)^p=1 \implies p=1<c$$
So $T_2(n)\in\Theta(n^{3/2})$.\\
Since $T_1\in\Theta(n\log_2n)$ \& $T_2(n)\in\Theta(n^{3/2})$ then $T_1$ scales better than $T_2$.\\
I would recommend using $T_1$.
}

\question{Asymptotic Behaviour}{
For each of these recurrences, given a function $f(n)$ such that $T(n)\in\Theta(f(n))$.
}

\qpart{
$T(n)=27T(\lfloor\frac{n}{3}\rfloor)+n$.
}

\spart{
Analysing with \textit{Master Theorem}
$$a=27,\ b=3,\ b\in O(n)\implies c=1$$
Let $p=\log_ba=log_327=3\implies c<p$.\\
Thus $T(n)\in\Theta(n^3)$.
}

\qpart{
$T(n)=2T(\lfloor\frac{n}{4}\rfloor)+T(\lceil\frac{n}{4}\rceil)+15n^2$.
}

\spart{
Analysing with \textit{Akra-Bazzi Formula}
$$k=2,\ a_1=2,\ a_2=1,\ d_1=\frac{1}{4},\ d_2=\frac{1}{4},\ \alpha=15,\ c=2$$
This obeys all conditions of \textit{Akra-Bazzi Formula}.\\
Set
\[\begin{array}{rcrcl}
\sum_{i=1}^2a_id_i^p=1 &\implies& 2(\frac{1}{4})^p+(\frac{1}{4}^p)&=&1\\
&\implies&(\frac{1}{4})^p&=&\frac{1}{3}\\
&\implies&p<1<2&=&2\\
&\implies& p&<&c
\end{array}\]
Thus $T(n)\in\Theta(n^2)$.
}

\qpart{
$T(n)=\frac{1}{2}T(\lfloor\frac{n}{4}\rfloor)+\frac{1}{3}T(\lfloor\frac{n}{3}\rfloor)+\frac{1}{6}T(\lceil\frac{n}{6}\rceil)+6$
}

\spart{
Analysing with \textit{Akra-Bazzi Formula}
$$k=3,\ a_1=\frac{1}{2},\ a_2=\frac{1}{3},\ a_3=\frac{1}{6},\ d_1=\frac{1}{3},\ d_2=\frac{1}{3},\ d_3=\frac{1}{6},\ \alpha=6,\ c=0$$
This obeys all conditions of \textit{Akra-Bazzi Formula}.\\
Set
\[\begin{array}{rcrcl}
\sum_{i=1}^2a_id_i^p=1 &\implies& \frac{1}{2}(\frac{1}{4})^p+\frac{1}{3}(\frac{1}{3})^p+\frac{1}{6}(\frac{1}{6})^p&=&1\\
&\implies&p&=&0\\
&\implies&p&=&c
\end{array}\]
Thus $T(n)\in\Theta(n^0\log_2n)\equiv T(n)\in\Theta(\log_2n)$.
}

\qpart{
$T(n)=3T(\lceil\frac{n}{2}\rceil)+T(\lfloor\frac{n}{5}\rfloor)+n^2$
}

\spart{
Analysing with \textit{Akra-Bazzi Formula}
$$k=2,\ a_1=3,\ a_2=1,\ d_1=\frac{1}{2},\ d_2=\frac{1}{5},\ \alpha=1,\ c=2$$
This obeys all conditions of \textit{Akra-Bazzi Formula}.\\
Set
\[\begin{array}{rcrcl}
\sum_{i=1}^2a_id_i^p=1 &\implies& 3(\frac{1}{2})^p+(\frac{1}{5})^p=1\\
\mathrm{Suppose\ }p=2 &\implies& 3(\frac{1}{4})+\frac{1}{25}=\frac{79}{100}&<&1\\
\mathrm{Suppose\ }p=0 &\implies& 3+1=4\\
&\implies& 0<p<2&=&c\\
&\implies&p&<&c
\end{array}\]
Thus $T(n)\in\Theta(n^2)$.
}

\question{Why five and not three?}{
Let A be a numerical array all of whose elements are different. Consider the SELECT algorithm from lectures.\\
Consider the algorithm SELECT-3 which does exactly the same thing as SELECT, except that instead of dividing the n elements into groups of 5, it divides
them into groups of 3. Suppose $n = 6m + 1$ for some integer $m \geq 0$, and the
time to divide n elements into groups of 3 and find the median of each of these
groups is an for some real number $a > 0$.
Show that there may be $4m + 1 = \lceil\frac{2n}{3}\rceil$ elements of $A[p,\dots,q]$ that are strictly
greater than $x$.\\
Show that there some $a_2 > 0$ and a lower bound $L(n)$ on the worst-case time to
run SELECT-3 on input size $n$ which satisfies $L(n) = L(\lceil\frac{2n}{3}\rceil)+a_2L(\lceil\frac{n}{3}\rceil)+
an$.\\
Show that $L(n) \in \Theta(n \log_2 n)$. Deduce that SELECT-3 is not $O(n)$.
}

\spart{
Let $n=6m+1$.\\
There are $2m$ full groups \& one group with $1$ element.\\
If $x$ is in the solo group then it is the median of its group.
Then there are $m$ groups with medians less than $x$, each of which can have only $1$ element greater than $x$.\\
And there are $m$ groups with medians greater than $x$, each of these can have $3$ elements greater than $x$.\\
This means there are at most $1(m)+3(m)=4m$ elements greater than $x$.\\
}

\spart{
%TODO
}

\question{Quick question on $\Theta$}{
Show that if $f(n)\in O(g_1(n)),\ g_1(n)\in\Theta(j(n))$ and $g_2(n) \in \Theta(j(n))$ then $f(n)+g_2(n)\in\Theta(j(n))$.\\
}

\solution{
Since $f(n)\in O(g_1(n))\ \exists\ n_0\in\nats,\ c\in\real^{>0}\ st\ 0\leq f(n) \leq cg_1(n)\ \forall\ b\geq n_0$.\\
And, since $g_1, g_2\in\Theta(j(n))\ \exists\ n_1\in\nats,\ d_1,d_2\in\real^{>0}\ st\ 0<d_1j(n)\leq g_1(n), g_2(n) \leq d_2j(n)$.\\
Then
\[\begin{array}{rcccl}
d_1j(n)&\leq&g_2(n)+f(n)&\leq& cg_1(n)+d_2j(n)\\
&&&\leq&d_2(1+c)j(n)\\
d_2(1+c)>0 &\implies& f(n)+g_2(n) &\in&\Theta(j(n))
\end{array}\]
}

\question{Seven's all right}{
As in question 5, let A be an array all of whose elements are different. Con-
sider the algorithm SELECT-7, which does exactly the same thing as SELECT (shown in question 5), except that instead of dividing the n elements into groups
of 5, it divides them into groups of 7 (except that the last group may have $\leq 7$
elements). Show that there are at most $n-4(\frac{1}{2}\lceil\frac{n}{7}\rceil-2)$ elements of $A$ less than the pivot $x$, and at most $n-4(\frac{1}{2}\lceil\frac{n}{7}\rceil-2)$ elements greater than thepivot $x$. Show that there is some $n_0$ such that $n-2(\lceil\frac{n}{7}\rceil-2) \leq\lfloor\frac{11n}{14}\rfloor\ \forall\ n>n_0$.\\
Use this and either strong induction, or the result shown in question 3 from the Akra-Bazzi forumla, to show that SELECT-7 is $O(n)$.
}

\spart{
There are $3$ elements in $x$'s group which are less than $x$.\\ %NOTE this is never used
There are $\frac{1}{2}\lceil\frac{n}{7}\rceil-2$ full groups with medians greater than $x$.\\
Each of these has at least $4$ elemetns greater than $x$.\\
So there are at least $4(\frac{1}{2}\lceil\frac{n}{7}\rceil-2)$ elements greater than $x$.\\
So there are at most $n-4(\frac{1}{2}\lceil\frac{n}{7}\rceil-2)$ elements less than $x$.
}

\spart{
There are $3$ elements in $x$'s group which are greater than $x$.\\ %NOTE this is never used
There are $\frac{1}{2}\lceil\frac{n}{7}\rceil-2$ full groups with medians less than $x$.\\
Each of these has at least $4$ elements less than $x$.\\
So at least $4(\frac{1}{2}\lceil\frac{n}{7}\rceil-2)$ elements are less than $x$.\\
So there are at most $n-4(\frac{1}{2}\lceil\frac{n}{7}\rceil-2)$ greater than $x$.
}

\spart{
\[\begin{array}{rcl}
n-2(\lceil\frac{n}{7}\rceil-2)&\leq&n-2((\frac{n}{7})-2)\\
&=&\frac{5n}{7}+4\ \forall\ n\in\nats
\end{array}\]
$n-2(\lceil\frac{n}{7}\rceil-2)$ is bounded above by $\frac{5n}{7}+4$.
\[\begin{array}{rcl}
\lfloor\frac{11n}{14}\rfloor&\geq&\frac{11x}{14}-1
\end{array}\]
$\lfloor\frac{11n}{14}\rfloor$is bounded below by $\frac{11n}{14}-1$.
\[\begin{array}{rrcl}
\mathrm{Set}&\frac{11n}{14}-1&\geq&\frac{5n}{7}+4\\
\implies&\frac{n}{14}&\geq&5\\
\implies&n&\geq&70
\end{array}\]
$n-2(\lceil\frac{n}{7}\rceil-2)$ is bounded above by $\frac{11n}{14}-1$.
Thus $\exists\ n_0\in\nats$ st $n-2(\lceil\frac{n}{7}\rceil-2)$ is bounded above \& $\lfloor\frac{11n}{14}\rfloor$ is bounded below by $\frac{11n}{14}-1$, thus $n-2(\lceil\frac{n}{7}\rceil-2) \leq \lfloor\frac{11n}{14}\rfloor\ \forall\ n >n_0$.
}

\end{document}
